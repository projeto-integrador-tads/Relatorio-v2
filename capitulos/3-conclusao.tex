\chapter{CONSIDERAÇÕES FINAIS}
% ---

A mobilidade entre as cidades do Vale do Jequitinhonha, especialmente no contexto do atual ciclo de extração de lítio, apresenta desafios significativos devido à escassez de veículos e à oferta limitada de serviços de transporte. A análise da frota de veículos e a comparação com o tamanho da população evidenciam a necessidade de soluções práticas para facilitar o deslocamento entre os municípios.

Além disso, a desigualdade na oferta de serviços em cada cidade intensifica as migrações pendulares, o que sobrecarrega as opções de transporte disponíveis atualmente. Nesse cenário, a implementação de um aplicativo de caronas pode ser uma solução inovadora, capaz de formalizar uma prática já existente, porém de forma mais segura e eficiente. Tal proposta visa não apenas garantir maior acessibilidade ao transporte, mas também promover preços justos e aumentar a oferta de viagens, contribuindo para o desenvolvimento socioeconômico da região.

Durante o desenvolvimento do \textit{back-end} foi possível criar vários \textit{endpoints} importantes para garantir um funcionamento inicial adequado, o que facilitou integrações com o \textit{frontend} futuramente. 

Poderão ser implementadas novas funcionalidades no \textit{back-end}, como integração com serviços de pagamento e validação de documentos. O projeto será desenvolvido com foco em escalabilidade, segurança de dados e alta disponibilidade, assegurando uma experiência fluida e segura tanto para motoristas quanto para passageiros.

As funcionalidades de criação de conta, cadastro de motorista, autenticação de usuário e reserva de carona estão plenamente funcionais, além de seguirem o conceito do \textit{Create, Read, Update, Delete (CRUD)}, essencial na criação de sistemas. 

A escolha do \textit{React Native} para o \textit{front-end} possibilitou a criação de uma versão inicial de aplicativo mobile multiplataforma com uma interface intuitiva e estável. Elementos visuais foram escolhidos visando a usabilidade e acessibilidade do projeto, criando uma identidade visual bem definida que busca a melhor experiência de uso, seja para passageiro ou motorista.

Por fim, a nova estruturação do projeto buscou a integração bem definida entre o \textit{front-end} e \textit{back-end}, permitiu uma demonstração mais clara do escopo do projeto. Com isso, espera-se que o aplicativo atenda às necessidades dos usuários e ofereça uma opção de serviço eficiente e seguro, promovendo um modelo de mobilidade compartilhada acessível e sustentável na região.
